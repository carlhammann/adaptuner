\documentclass{scrartcl}

\usepackage[utf8]{inputenc}
\usepackage[english]{babel}
\usepackage[autostyle]{csquotes}
\usepackage{amsmath, amssymb}
\usepackage{hyperref}

\begin{document}

Imagine a chord as a system of nodes (the notes of the chord) connected by
springs (the intervals). The relaxed length of each of the springs corresponds
to the desired size of its interval.

In the best of all possible cases, the lowest energy configuration of the
system has each spring at its relaxed length: Every interval has its desired
size, the chord is \enquote{in tune}. Otherwise, the lowest energy
configuration describes some tempered version of the chord, where the amount by
which the individual intervals are out of tune is negotiated by the how stiff
the springs are. For example, we can make fifths harder to detune than thirds
by making their springs stiffer. In section \ref{sec:rods}, I also introduce
rods to the system, which correspond to intervals that may never be detuned
(octaves, for example?).

Also, in section \ref{sec:fixedsprings}, I describe how to add springs to a
fixed point. This can be used to anchor the chord to any pitch reference --
some pre-defined scale, or previously sounding notes, for example.

\section{Nodes connected by springs}

Let's start with a system of nodes connected by springs.
\begin{enumerate}
  \item The nodes are constrained so that they can only move on one line.
    Denote the positions of the nodes by \(x_1,\cdots,x_n\).
  \item For \(i<j\), there is a spring from node \(i\) to node \(j\).
    Let \(l_{ij}\) denote its relaxed length, and \(a_{ij}\ge0\) its stiffness.
\end{enumerate}
If we let the system move freely, it will tend to the state of lowest
energy. The energy in all the tensioned springs is given by
\begin{equation}\label{eq:energy}
  E(x):=\sum_{i<j}a_{ij}(x_j-x_i-l_{ij})^2\ .
\end{equation}
Expanding this as
\[
  E(x):=\sum_{i<j}a_{ij}(x_j^2+x_i^2+l_{ij}^2-2x_jx_i-2x_jl_{ij}+2x_il_{ij})\ ,
\]
we see that \(E\) can be written as \(E(x)=x\cdot Ax+b\cdot x + c\), where
\begin{gather*}
  A_{ij}:=
  \begin{cases}
    -a_{ij}\ ,& i<j\\
    -a_{ji}\ ,& i>j\\
    \sum_{i<k}a_{ik}+\sum_{i>k}a_{ki}\ ,& i=j
  \end{cases}\ ,\\
  b_i:=-2\sum_{i>j}a_{ji}l_{ji}+2\sum_{i<j}a_{ij}l_{ij}\ ,\\
  c:=\sum_{i<j}a_{ij}l_{ij}^2\ .
\end{gather*}
Now, a condition for the \(x\) with minimal energy is \(\nabla
E(x)=0\), and since
\[
  \nabla E(x) = Ax + A^\top x + b = 2Ax + b\ ,
\]
finding the minimum amounts to solving the linear system of equations
\begin{equation}\label{eq:mincond}
  -Ax=\frac{1}{2}b\ .
\end{equation}
However, \(A\) is not invertible (to see this, note that its column
sum is zero). This makes intuitive sense: The relative positions of
the nodes are determined by the springs, but their absolute positions
\(x_i\) are not yet determined. I'll address this problem in section
\ref{sec:fixedsprings}

\section{Interlude: zero forces}\label{sec:zeroforce}

Let's take another look at the linear system \eqref{eq:mincond}.
The \(i\)-th row of this system says
\[
  -x_i\left(\sum_{i<k}a_{ik}+\sum_{i>k}a_{ki}\right)+\sum_{i<j}a_{ij}x_j+\sum_{i>j}a_{ji}x_j
  =
  -\sum_{i>j}a_{ji}l_{ji}+\sum_{i<j}a_{ij}l_{ij}\ ,
\]
Reordered, this is
\[
  0=
  \sum_{i<j}a_{ij}(x_j-x_i-l_{ij})
  + \sum_{i>j}a_{ji}(x_j-x_i+l_{ji})\ .
\]
The first sum (over \(i<j\)) describes the forces on node \(i\) due to
springs \enquote{from the right}: Look at the summand
\(a_{ij}(x_j-x_i-l_{ij})\)
\begin{itemize}
  \item if \(x_i>x_j-l_{ij}\), the summand will be negative, which
    describes the force pushing the node to the left.
  \item if \(x_i<x_j-l_{ij}\), the summand will be positive,
    describing a force pulling the node to the right.
\end{itemize}
Similarly, the second sum describes the forces exerted on node \(i\)
by springs from the left.

In summary, the energy minimality condition \eqref{eq:mincond} has a
physical interpretation: The sum of forces at each node must be zero.
This makes intuitive sense. Thus, another way to frame the problem is
to look at the vector of forces \(f:=-Ax-\frac{1}{2}b\) at each node and
require \(f=0\).

\section{Adding springs from a fixed point}\label{sec:fixedsprings}

Let's modify our energy \eqref{eq:energy} by adding some springs that
start at the origin and end at one of the nodes:
\[
  E(x):=
  \sum_{i<j}a_{ij}(x_j-x_i-l_{ij})^2
  +\sum_ib_i(x_i-p_i)^2
  \ .
\]
Then, we can write \(E(x)=x\cdot Ax+b\cdot x + c\) as above, with
\begin{gather*}
  A_{ij}:=
  \begin{cases}
    -a_{ij}\ ,& i<j\\
    -a_{ji}\ ,& i>j\\
    b_i+\sum_{i<k}a_{ik}+\sum_{i>k}a_{ki}\ ,& i=j
  \end{cases}\ ,\\
  b_i:=-2b_ip_i-2\sum_{i>j}a_{ji}l_{ji}+2\sum_{i<j}a_{ij}l_{ij}\ ,\\
  c:=b_ip_i^2+\sum_{i<j}a_{ij}l_{ij}^2\ .
\end{gather*}
Now, \(A\) can have full rank, and the system \eqref{eq:mincond} can
have a unique solution.

\section{Adding rods}\label{sec:rods}

First, let's simplify notation: In the scenario discussed so far,
everything boils down to solving a linear system of equations
\(Ax=b\). In the setting of section \ref{sec:fixedsprings}, the
system is given by
\begin{gather*}
  A_{ij}:=
  \begin{cases}
    a_{ij}\ ,& i<j\\
    a_{ji}\ ,& i>j\\
    -b_i-\sum_{i<k}a_{ik}-\sum_{i>k}a_{ki}\ ,& i=j
  \end{cases}\ ,\\
  b_i:=-b_ip_i-\sum_{i>j}a_{ji}l_{ji}+\sum_{i<j}a_{ij}l_{ij}\ .
\end{gather*}
(Note the changed sign/constant factor compared to section
  \ref{sec:fixedsprings}! This is due to the form of equation
\eqref{eq:mincond}.)

Assume now that some of the nodes are connected not by springs but by
(incompressible, non-stretchable) rods. The forces at both ends of
each rod are the same; each spring attached to one end can be felt at
both ends. This suggests a way to transform the system of equations,
adopting the viewpoint of section \ref{sec:zeroforce}, where we noted
that row \(i\) of the system encodes the condition that the force at
node \(i\) is zero:

We'll construct a system \(\widetilde Ax=\widetilde b\) from \(Ax=b\) by the
following method. In order to add a rod from node \(i\) to node
\(j\):
\begin{enumerate}
  \item The force at node \(i\) in the new system must be the sum of
    forces at nodes \(i\) and \(j\) in the old system, and we still
    require that it be zero. Thus, the \(i\)-th row of \(\widetilde A\)
    must be the sum of the \(i\)-th and \(j\)-th rows of \(A\), and
    \(\widetilde{b_i}:=b_i+b_j\).
  \item The same could be said about the force at node \(j\), but
    setting the \(j\)-th row of \(\widetilde A\) to the same sum as
    the \(i\)-th row would create an underdetermined system. The
    information missing in that system is the length of the rod. So,
    let's use the \(j\)-th row to encode the relation
    \(x_j-x_i=l_{ij}\), where we reuse the symbol \(l_{ij}\) to
    denote the length of the rod from \(i\) to \(j\). That is,
    \(b_j:=l_{ij}\), \(\widetilde{A_{ji}}:=-1\), and \(\widetilde{A_{jj}}:=1\).
\end{enumerate}
Applying these two steps has the important side-effect that we
delete force information at node \(j\). This entails two limitations
when applying them iteratively to many pairs \(i,j\): Once node \(j\)
has been used as the end point of a rod
\begin{itemize}
  \item we cannot use it as the end point of another rod, and
  \item we cannot use it as the start point of another rod.
\end{itemize}
At first, these might seem serious, but note that
\begin{itemize}
  \item We don't need \enquote{chains} of rods, since they can be
    substituted by rods to the lowest node in the chain.
  \item We don't need triangles of rods, since they'll have to be
    degenerate, as we require all nodes to lie on one line. Thus, one
    rod can always be deleted without loss of information.
\end{itemize}
So, given any specification of how nodes should be connected by rods,
we can obtain an equivalent specification that has one or more
\enquote{base} nodes, from which rods go out other, non-base, nodes,
and no non-base node is connected to more than one base node.

\section{Calculating with rational matrices}

Let's work with the system \(\widetilde Ax=\widetilde b\) from
section \ref{sec:rods}, dropping the tildes. For the application of
interest, we know more:
\begin{enumerate}
  \item \label{it:rationalstiffness} The
    stiffnesses \(a_{ij}\) and \(b_i\) are nonnegative
    rational numbers.
  \item \label{it:lengthlinearcomb} There exist
    \(r_1,\ldots,r_m\in\mathbb{R}\) such that the \(l_{ij}\) and
    \(p_i\) can be written as rational linear combinations of \(r_i\).
  \item
    We're interested in expressing the solution \(x_i\) as rational
    linear combinations of the \(r_i\).
\end{enumerate} Fix an arbitrary vector
\(h\) containing all values that the \(l_{ij}\) or the \(p_i\) take. That is:
\begin{itemize}
  \item There is a function \(\lambda\) such that for all
    \(i<j\), \(h_{\lambda(i,j)}=l_{ij}\) and a function
    \(\pi\) such that for all \(i\), \(h_{\pi(i)}=p_i\).
  \item  There is a rational matrix \(L\) such that \(h=Lr\), by item
    \ref{it:lengthlinearcomb}.
\end{itemize} Further, there is a matrix \(B\) such that \(b=Bh\), given by
\begin{align*} B_{i,\pi(i)}&:=-b_i\\
  B_{i,\lambda(i,j)}&:=a_{ij}&\text{if } i<j\\
  B_{i,\lambda(i,j)}&:=-a_{ji}&\text{if } i>j\\ B_{i,j}&:=0&\text{ everywhere
  else}
\end{align*}

We can now calculate \(x=A^{-1}BLr\). That is to say: the rows of \(A^{-1}BL\)
hold the coefficients of the rational linear combinations we desire.
(Everything is rational, because item \ref{it:rationalstiffness} above
means that \(A\) and \(B\) are rational matrices.)

\end{document}

% vi: set spell spelllang=en:
