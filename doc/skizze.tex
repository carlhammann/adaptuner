\documentclass{scrartcl}

\usepackage[utf8]{inputenc}
\usepackage[T1]{fontenc}
\usepackage{microtype}

\usepackage[ngerman]{babel}
\usepackage[german=quotes]{csquotes}

\usepackage[titletoc]{appendix}

\usepackage{hyperref}

\title{Projektskizzen für das adaptiv gestimmte Klavier}
\author{Carl Hammann}
\date{Version vom 11. Juni 2024}

\begin{document}

\maketitle

\tableofcontents


\vspace{1cm} 

\noindent Der folgende Text gibt einen Überblick über den momentanen Stand
meiner Überlegungen und über aus meiner Sicht interessante
Forschungsprojekte. Ich sehe die Projektideen nicht als
vollständig festgelegt oder klar voneinander getrennt; es geht mir eher
darum, ein Spektrum zur Diskussionsgrundlage zu öffnen.

Für diese informelle Diskussion verzichte ich auf
eine wissenschaftliche Zitierweise, auch, weil ich zum Teil nicht genug
Überblick über die Literatur habe.

\section{Motivation: Kontrollierte Abweichungen von der reinen Stimmung}

In den für \enquote{die} westliche Kunstmusik relevanten Systemen der reinen
Stimmung gibt es -- mindestens konzeptuell -- unendlich viele
verschiedene Intervalle innerhalb einer Oktave. Das ist einerseits für
aufführende Musiker:innen und Instrumentenbauer:innen unpraktisch und
andererseits eine theoretische Herausforderung: Die meiste Musik
ist nicht explizit in reiner Stimmung konzipiert. Selbst in
Stilen, in denen das Ideal der reinen Stimmung große Relevanz hat, gibt
es oft keine offensichtlich korrekte Intonation. Das Repertoire ist
voller Spannungen zwischen \enquote{Kommafallen} und ungewohnten melodischen
Intervallen sowie voller Akkorde, die notwendigerweise mindestens ein
\enquote{verstimmtes}
Intervall enthalten. 

Temperierte Stimmungen umgehen die theoretische und praktische
Unhandlichkeit der reinen Stimmung(en), indem sie endlich viele Tonhöhen
pro Oktave fixieren. Freilich zahlen sie damit den Preis, dass sie nur
mehr oder weniger gute Vertreter der reinen Intervalle beinhalten.
Historisch führte der Weg immer besserer Approximationen zu Systemen mit
19, 31, 55... Tönen pro Oktave, die progressiv immer bessere Quinten und
Terzen erlauben.

Demgegenüber steht die Praxis eines Chores oder Streicherensembles:
Intonation bedeutet hier nicht, fixierte Tonhöhen zu reproduzieren,
sondern besteht in einer Verhandlung zwischen den harmonischen und
melodischen Intervallen, die sich beide an der reinen Stimmung
orientieren, aber situativ von ihr abweichen. Diese Perspektive
legt die Idee eines adaptiv gestimmten, mehrstimmig spielbaren
Instruments nahe, das einen Experimentiergrund zur Erforschung eben
dieser Abweichungen bietet.

Welche Strategien beispielweise ein moderner Chor
anwendet, kann man untersuchen, indem man alle Sänger:innen
mikrofoniert und genaue Tonhöhen misst. Soweit ich weiß, gibt es ein
kleine Anzahl solcher Studien, die allerdings keine
abschließenden Aussagen machen. Diesem analytischen Ansatz ist ein
Instrument, wie ich es vorschlage, das synthetische Gegenstück.

In den letzten Monaten habe ich Prototypen von Computerprogrammen
entwickelt, die diesen Pfad betreten: Ein e-Piano wird in wenigen
Millisekunden nach vorprogrammierten oder von dem:der Spieler:in zu
beeinflussenden Heuristiken umgestimmt. (In den Anhängen gehe ich auf
den momentanen Stand des Prototypen ein sowie auf die zwei mir bekannten
ähnlichen Projekte.) Jetzt ist aus meiner Sicht der Punkt erreicht, an
dem ich in der Lage bin, interessante Fragen zu formulieren, von denen
jede ein musikpratisches oder -theoretisches Forschungsprojekt variablen
Umfangs werden kann.

Meine Zielgruppe sind alle Musiker:innen, die sich für reine Intonation
interessieren und ein e-Piano haben, aber insbesondere Chorsänger:innen
und -leiter:innen.

\section{\enquote{nulltes} Projekt: Weiterentwicklung des Instruments an sich}

Jede Weiterarbeit setzt eine entsprechende Weiterentwicklung des
Computerprogramms voraus. Dafür ist aber aus Sicht der
Programmierung keine neuartige Arbeit notwendig, daher sehe ich diesen
Teil auch nicht als einen Forschungsgegenstand. Weitere
Programmierarbeit wird notwendig sein, aber habe ich das Gefühl, die
rein technischen Probleme im Prinzip gelöst zu haben.

Interessanter sind Fragen dazu, welche Metaphern, Visualisierungen und
Bedienelemente das Programm verwenden soll, um für Musiker:innen ein
sinnvolles Werkzeug zu werden. Was \enquote{sinnvoll} ist, hängt von dem
weiter zu beschreitenden Pfad ab. Die Anhänge enthalten Bemerkungen zu
meinem momentanen Stand.

Ich sehe ein musikalisch sinnvoll benutzbares Instrument als eines der
Hauptziele jedes der weiteren Projekte an; nur der Fokus wird jeweils
ein anderer sein.

\section{erstes Projekt: Probenwerkzeug für Chöre}

Wie erhält man ein effektives Probenwerkzeug für Chöre? Dies ist eine
Frage dazu, wie das Programm weiter zu entwickeln wäre, um es leichter
spiel- und verstehbar zu machen, aber auch eine probenmethodische
Untersuchung. Meine Idee ist, dass man singen kann, was man sich
vorstellen kann, und sich vorstellen kann, was man gehört hat. Kann man
vielleicht sogar einmal \enquote{absichtlich} in eine Kommafalle tappen
und erfahren, wie sich das anfühlt? 

Hat solches Wissen um Intonation außerhalb von (spezialisierten)
(Profi-) Ensembles praktische Relevanz? Wie (und wem?) lehrt man es
mithilfe des zu entwickelnden Instruments?

\section{zweites Projekt: Was sagen die alten Quellen?}

Das Ideal der reinen Stimmung ist in der Renaissance- und
Frühbarockmusik relevanter, als es zum Beispiel heute ist. Was sagen
Theoretiker (der Renaissance) zu dem Thema der Abweichungen von diesem
Ideal und ist das, was sie sagen, explizit genug, um Regeln abzuleiten,
die sich in das Instrument programmieren lassen? Falls ja: Klingt das
Ergebnis plausibel, wenn man es spielt? Kann man es mit einem Ensemble
umsetzen?

\section{drittes Projekt: Neue Musik}

Welche Möglichkeiten bieten adaptive Stimmungen für neue Musik? Das ist
einerseits eine weitere Frage zur Probenmethodik: Gibt es eine
Anwendung für Proben von Musik, die explizit in reiner Stimmung
geschrieben wurde? Was bräuchte es zum Beispiel um die Streichquartette
von Ben Jonhston auf dem e-Piano zu spielen, und ist das ein hilfreiches
Werkzeug in Proben dieser Musik?

Andererseits gibt es hier auch eine Frage nach neuen kompositorischen
Möglichkeiten, die explizit die Abweichungen von der reinen Stimmung in
den Blick nehmen. Zu diesem Aspekt habe ich zwei Ideen, die insbesondere
auch der Auslöser meiner Beschäftigung mit adaptiver Stimmung waren:

\begin{itemize}
\item In nicht-oktavperiodischen Stimmungen
  gibt es auch Kommaprobleme. Ich vermute, dass einer der Gründe, warum
  Musik in solchen Systemen schwierig zu singen ist, darin liegt, dass
  wir nicht wissen, wie eine natürliche Intonation
  klingt. So kommen zusätzlich zu den ungewohnten Harmonien auch noch
  die Probleme, die man sich einhandelt, wenn man entweder in reiner
  Stimmung arbeitet (dann hat man schwierige melodische Intervalle) oder
  in einer schlechten temperierten Stimmung (dann hat man keine echten
  Konsonanzen und die Musik, die sowieso schon wegen ihrer
  Ungewohntheit \enquote{verstimmt} klingt, ist tatsächlich verstimmt).
\item Auch in \enquote{normal gestimmter} Musik intonieren wir nicht alle Akkorde
  rein (das ist schließlich der Ausgangspunkt dieser Beschäftigung),
sondern rasten nur zu \enquote{wichtigen} Akkorden in die --
  durch das Phänomen der Obertöne leicht zu findende -- reine Stimmung
  ein. Könnte ein Verständnis, wie dieses Einrasten funktioniert, es
  ermöglichen, ein live singbares Stück zu schreiben, das
  in einen ungewöhnlichen -- aber ebenfalls physikalisch
  \enquote{legitimierten}  -- Akkord einrastet, zum Beispiel den eines Gongs?
\end{itemize}

\section{viertes Projekt: \enquote{beste} Annäherungen an die reine Stimmung}

Alle bislang beschriebenen Projekte gehen davon aus, dass der:die Spieler:in weiß, was
er:sie tut. Insbesondere lassen die \enquote{Orgelregister für
Intonation}
(siehe Anhang \ref{sec:register}) die Kontrolle bei dem:der Spieler:in. Meine bislang
unausgesprochene Vermutung ist, dass ein sich selbstständig
verstimmendes Klavier noch verwirrender ist als eines, das sich
wenigstens nach von mir gewählten, einfachen Regeln verstimmt.

Ein anderer Ansatz, der ein Forschungsprojekt mit größerem
Mathematikanteil nach sich
zöge,
wäre zu versuchen, rigorose Definitionen zu formulieren, die (bestimmte
Aspekte davon) einfangen, was es intuitiv heißt, dass ein Akkord/eine
Akkordverbindung/eine polyphone Struktur rein gestimmt ist. Davon ausgehend könnten Optima dieser Zielfunktionen auf verschiedenen Korpora
von Musik gesucht werden. Lassen sich daraus Regeln ableiten,
welche die Möglichkeiten der adaptiven Stimmung nutzen, um Musik
\enquote{reiner} zu stimmen als mit herkömmlichen Verfahren? Können solche Regeln
Inspiration für neue \enquote{Intonationsregister} werden? Es gibt eine
ganze Menge Literatur zu diesem Thema, über die ich aber noch keinen
Überblick habe. Oft scheint die Zielsetzung zu sein,
Toningenieuren ein Werkzeug zu geben, schlechte Intonation im Nachhinein zu reparieren.

Der Vollständigkeit halber möchte ich erwähnen, dass es bestimmt
ein förderbares Projekt wäre, KI-Systeme darauf zu trainieren, die
Entscheidungen zu fällen, die sonst der:die Spieler:in zu machen hat.
So könnte beispielsweise eine \enquote{Intonations-KI} trainiert werden, die das Gesamtwerk von Schütz
\enquote{besser} intoniert als jeder menschliche Chor. Ich erwähne diese
Forschungsrichtung vor dem Hintergrund des momentanen KI-Hypes; mein persönliches
Interesse liegt eher darin, mehr über Intonation zu lernen, und
Musiker:innen ein neues, absichtsvoll nutzbares Instrument zu bauen.

\clearpage
\centerline{\textsf{\textbf{Anhang}}}

\begin{appendices}

\section{\enquote{Orgelregister für Intonation}}\label{sec:register}

Die meiner Meinung nach fruchtbarste Metapher, die ich gerade verwende,
ist die eines Orgelregisters für Intonation. Je nach Situation
entscheidet man, wie zu intonieren ist, indem man Register an- und
abwählt.

Momentan teste ich zwei Typen von Registern: Solche zur Auswahl
des als \enquote{rein} betrachteten Tonvorrats und solche, die bestimmte
\enquote{Basisintervalle} gezielt verstimmen. 

In der folgenden Erklärung beschränke ich mich auf den Fall, in dem es
zwölf Töne pro Oktave gibt und in dem die Basisintervalle die Oktaven,
Quinten und großen Terzen sind. (Beide Einschränkungen sind nicht
designbedingt. Im Prinzip erlaube ich auch Stimmungssysteme, die eine
andere Periodizität als die Oktave -- oder gar keine -- haben und die
beliebig viele Basisintervalle, zum Beispiel für höhere Naturtöne,
beinhalten.)

Die Register des ersten Typs, zur Auswahl des reinen Tonvorrats, haben
zwei Aufgaben:

\begin{itemize}
\item Sie legen die die harmonischen Intervalle fest. (Beispiel: Welches A
  gehört über einen C-Dur Akkord? -- Das einen großen Ganzton über G,
  oder das einen kleinen Ganzton über G?)
\item Sie wählen die Kandidaten für den \enquote{nächsten Referenzton}. (Beispiel:
  Gerade spiele ich C-Dur, will ich danach Fis-Dur oder Ges-Dur?)
\end{itemize}

Die Register vom zweiten Typ, welche die Basisintervalle verstimmen,
können eingesetzt werden, um Kommafallen auszuweichen: Beispielsweise
wird in einer Quintfallsequenz A-D-G-C-F aus reinen Quinten die Terz des
finalen F-Dur um ein syntonisches Komma tiefer als der Grundton des
A-Dur am Anfang werden. Das ist unter Umständen nicht erwünscht. Einen
möglichen Ausweg bietet ein Register, das Quinten um ein Viertelkomma zu
klein macht (d.h. mitteltönige Quinten für eine begrenzte Zeit und
ab einem bestimmten Bezugston). Die Register dieses Typs können entweder
nur auf die Folge der Referenztöne angewendet werden oder auch auf die
harmonischen Intervalle. Im Beispiel der Quintfallsequenz könnte man
also entweder nur die Bassmelodie mit mitteltönigen Quinten spielen,
während alle anderen Stimmen darüber rein intonieren, oder den ganzen
\enquote{Chor} für ein paar Akkorde mitteltönige Quinten auch als harmonische
Intervalle benutzen lassen.

\section{Visualisierungen}

Ein Klavier, das sich andauernd selbst verstimmt, ist sehr verwirrend
und braucht gute Visualisierungen.

Der aktuelle Prototyp zeigt ein Quinten-Terzen-Gitter, auf dem klingende
Noten und Kandidaten für nächste Referenztöne hervorgehoben werden. Wenn
ein \enquote{Verstimmungs}-Register (siehe vorheriger Abschnitt) benutzt wird, wird der Grad der Verstimmung
farblich veranschaulicht, zusätzlich zu einer Angabe in Cent.

Für die Notennamen finde ich die Notation von Ben Johnston
praktisch, denn sie kommt mit wenigen zusätzlichen Vorzeichen aus. Hier wird auch der pädagogische Aspekt berührt:
Mir ist wichtig, dass das Programm nicht nur undurchsichtig arbeitet, sondern auch
etwas lehrt. Durch die Benutzung eines Systems, in dem das syntonische Komma ein
Vorzeichen ist und in dem es keine enharmonischen Verwechselungen gibt,
werden Konzepte sichtbar.

Das Quinten-Terzen-Gitter hat mindestens seit Euler Tradition.
Visualisierungen, die auch noch auf höherdimensionale Systeme
(mehr Basisintervalle) verallgemeinert werden können, sind ein offenes
Forschungsfeld. So macht Robin Hayward sich mit seinem
\href{https://robinhayward.com/hayward-tuning-vine/}{\emph{Tuning
Vine}} zur
Visualisierung der Intervalle der reinen Stimmung tiefgreifende
Gedanken.

Zwei Forschungsfragen sind für mich besonders ersichtlich:

\begin{itemize}
\item Es gibt gute Notationen und Visualisierungen für die reine Stimmung.
  Wie veranschaulicht man die notwendigen Abweichungen?
\item Welche Informationen für ein spielbares Instrument hilfreich und mit
  wenig zusätzlichem Hardware-Aufwand realisierbar?
\end{itemize}

\section{Bereits bestehende, ähnliche Softwaresysteme}

Ich bin nicht der Erste, der die Idee für ein solches oder ähnliches
Computerprogramm hat. Mir sind zwei Projekte bekannt: 

\begin{itemize}
\item \href{https://mutabor.sourceforge.io/index.html.en}{\emph{Mutabor}} ist ein an der TU Dresden entwickeltes Programm, das auch
  Gegenstand einer kleinen Anzahl von Publikationen ist. Das Programm
  ist recht alt und in den letzten Jahren kaum gepflegt worden. (Das
  äußert sich insbesondere darin, dass es mir nicht gelingt, mit den mir
  zur Verfügung stehenden e-Pianos ein funktionierendes System zu
  starten.) Mutabor nähert sich dem Problemfeld adaptiver Stimmungen mit
  einer speziell für diesen Zweck entwickelten Programmiersprache. Diese
  Sprache ist auch der Fokus der Forschung zu Mutabor: Welche
  temporallogischen Konstrukte sind für das Problem relevant und wie
  operationalisiert man sie in einem Softwaresystem? Im Fokus steht die
  Forschung in der Logik und Informatik, adaptive Stimmungen dienen
  dabei als ein nicht-triviales Anwendungsgebiet. Mutabor stellt
  insbesondere nicht das Bedürfnis von Musiker:innen nach einem einfach
  spielbaren Instrument in den Vordergrund. Damit unterscheidet es sich
  von meinem Forschungsvorhaben.
\item \href{https://www.tallkite.com/alt-tuner.html}{\emph{alt-tuner}} ist ein offenbar ziemlich weit entwickeltes Hobbyprojekt,
  das aber weitgehend verlassen zu sein scheint. Es gibt keinen
  quelloffenen Code, den man inspizieren oder
  weiterentwickeln könnte. Ferner legt das Projekt das Augenmerk auf
  die Idee einer adaptiven reinen Stimmung, während mich eben die oben
  beschriebenen adaptiven \emph{Abweichungen} interessieren. Daher ist dieses
  Projekt für mein Forschungsvorhaben nicht relevant.
\end{itemize}

\end{appendices}

\end{document}

<!-- vi: spell spelllang=de textwidth=72
  -->
